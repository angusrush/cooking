\documentclass[a4paper,12pt]{scrreprt}

\usepackage[margin=1in]{geometry}
\usepackage{amsmath}
\usepackage{amsfonts}
\usepackage{amssymb}
\usepackage{amstext}
\usepackage{amsthm}
\usepackage{soul}
\usepackage{hyperref}
\usepackage{xcolor}
\usepackage{enumitem}
\usepackage{biblatex}
\usepackage{multicol}
\usepackage{booktabs}
\usepackage{appendix}
\usepackage{libertine}
\usepackage[libertine]{newtxmath}

% new page every section
\newcommand{\sectionbreak}{\clearpage}

%bibliography location
\addbibresource{/home/angus/latex/cooking/bibliography.bib}

% Set up hyperref colors
\hypersetup{colorlinks,
  linkcolor={blue!50!black},
  citecolor={blue!50!black},
  urlcolor={blue!80!black}
}

% Makes indentation more note-ish
\setlength{\parindent}{0em}
\setlength{\parskip}{.5em}
\setlength{\headheight}{14.0pt}

% My commands
\newcommand{\pder}[2]{\frac{\partial{} #1}{\partial{} #2}}
\newcommand{\tder}[2]{\frac{\text{d} #1}{\text{d} #2}}
\renewcommand{\d}{\mathrm{d}}
\newcommand{\R}{\mathbb{R}}
\newcommand{\C}{\mathbb{C}}
\newcommand{\Z}{\mathbb{Z}}
\newcommand{\Q}{\mathbb{Q}}
\newcommand{\N}{\mathbb{N}}
\newcommand{\dd}{\text{d}}
\newcommand{\Mod}[1]{\(\text{mod}#1\)}
\newcommand{\defn}[1]{\ul{#1}}
\newcommand{\abs}[1]{\left| #1 \right|}

% Second level of list uses empty bullets
\renewcommand{\labelitemii}{$\circ$}

% Declare theorem styles
\theoremstyle{definition}
\newtheorem{definition}{Definition}[section]
\newtheorem{example}{Example}[section]
\newtheorem{counterexample}{Counterexample}[section]
\newtheorem{tip}{Tip}[section]
\newtheorem*{ingredients}{Ingredients}

\theoremstyle{plain}
\newtheorem{theorem}{Theorem}[section]
\newtheorem{lemma}{Lemma}[section]
\newtheorem{corollary}{Corollary}[section]

\theoremstyle{remark}
\newtheorem{claim}{Claim}[section]
\newtheorem{recipe}{Recipe}[section]
\newtheorem{note}{Note}[section]
\newtheorem{notation}{Notation}[section]
\newtheorem{joke}{Joke}[section]

\newenvironment{directions}{\textit{Directions}.}

\title{My recipes}
\author{Angus Rush}

\begin{document}
\maketitle
\tableofcontents
\chapter{Introduction}

\section{What is this?}
\label{sec:what_is_this}

These are my personal notes.

\section{General information}\label{sec:general_information}

%\begin{tabular}{cc}
%  Food & Safe internal temperature \\
%  \midrule
%  Chicken & $75^{\circ}$C \\
%  Egg & $65^{\circ}$C
%\end{tabular}

\subsection{When should I salt?}
\label{ssc:when_salt}

Many recipes involving, say, sweating down vegetables, call for salt to be added in the early stages of a recipe. This is not because one wants to begin seasoning the food early for some reason, but because salt draws water out of the cells of vegetables via osmosis.

Osmosis is exemplified by the following process: two containers of water are separated by a membrane which allows water to pass through, and salt is added to one of the containers. If the membrane does not allow salt to pass through, then water will be drawn out of the container without salt into the container with salt with some pressure, known as \emph{osmotic pressure.}

In the case of cooking, the two containers are the inside and outside of a cell, and the membrane is the cell wall. Salting vegetables in the pan causes an osmotic pressure gradient across the cell wall, drawing water out of the cells and into the pan.

But what causes this osmotic pressure gradient? The short answer is that adding a solute to one of the containers increases the chemical potential in that container; it is then energetically favorable to to equalize the chemical potentials by diluting the container containing solute. However, this is not very intuitive for those who, like the author, lack a deep understanding of thermodynamics, and who want an elementary (or at least intuitive) explanation. Many elementary incorrect or partially correct explanations exist in the literature. For a review of this, see \cite{whatisosmosis}. A more complete derivation, starting from Newton's laws, is given in \hyperref[ch:the_virial_theorem]{Appendix~\ref*{ch:the_virial_theorem}}. The main source is \cite{physicsofosmoticpressure}.

There is a secondary effect of the addition of solutes such as salt or sugar to water: the boiling point of the water increases. This is of great importance in applications such as candy-making, where having great control over the temperature of sugar solutions becomes essential. This is due to a change in

\subsection{Harshness in alliums}
\label{ssc:harshness_in_alliums}

Plants in the genus \emph{allium,} such as onion, garlic, scallions, shallots, chives, etc., take sulferous compounds out of the earth and use them to create what Harold McGee calls \emph{ammunition compounds.} When the cell walls are broken, corresponding enzymes convert these ammunition compounds into harsh-tasting

contain an enzyme called \emph{alliinase.} When the cell walls of these plants are broken, this enzyme reacts with

Acid slows the reaction rate of alliinase into allicin. There appears to be conflicting information about whether allicin is itself responsible for the harsh-taste in garlic, or whether this comes from sulfur-containing compounds that the allicin decays into.


\chapter{Salad}
\section{Caesar salad dressing}
\begin{ingredients}
  $\,$
  \begin{multicols}{2}
    \begin{itemize}
      \item $1/2$ cup EVOO

      \item 2 medium cloves garlic

      \item 1 oz Parmesan, finely grated

      \item 1 egg yolk

      \item 1 Tbsp juice from 1 lemon

      \item 5 anchovies

      \item 1 tsp Worcestershire sauce

      \item White wine vinegar, to taste

      \item 1 tsp Dijon mustard (optional)
    \end{itemize}
  \end{multicols}
\end{ingredients}
\begin{directions}
  $\,$
  \begin{enumerate}
    \item Add lemon juice to a mortar and pestle. Add garlic and a pinch of salt, and mash to a paste.\footnote{We mash the garlic in an acidic environment because of REF.} Add anchovies and mash, then add Worcestershire sauce. Set aside

    \item Make a mayonnaise out of the egg yolk and olive oil, then whisk in the mixture from step 1. Season with salt, pepper, dijon mustard, and white wine vinegar to taste.
  \end{enumerate}
\end{directions}

\section{Tuna and thyme salad}
\label{sec:tuna_and_thyme_salad}

This makes enough salad for my lunch. Having said that, I eat quite a lot. The dressing is inspired by niçoise.

\begin{ingredients}
  $\,$
  \begin{multicols}{2}
    \begin{itemize}
      \item 1 medium clove garlic

      \item 1 anchovy

      \item Leaves from few sprigs of fresh thyme

      \item 1 small shallot, finely minced

      \item $1\frac{1}{2}$ Tbsp white wine vinegar

      \item 1 tsp Dijon mustard

      \item $\frac{1}{2}$ cup EVOO

      \item $1/2$ head lettuce, cut into salad-sized pieces, rinsed, and dried

      \item $1/2$ can tuna
    \end{itemize}
  \end{multicols}
\end{ingredients}
\begin{directions}
  $\,$
  \begin{enumerate}
    \item Add white wine vinegar to a mortar and pestle. Add garlic and a pinch of salt, and mash until the big pieces are broken down. Add anchovy and mash until a homogeneous consistency is reached.

    \item Transfer to a large bowl along shallot, thyme, and dijon mustard. Emulsify in EVOO. Salt and pepper to taste. Add lettuce, and toss to coat. Add tuna, break up as much as is desired, and gently toss to combine.
  \end{enumerate}
\end{directions}

\section{White bean and tuna salad}
From \cite{seriouseatswhitebeansalad}
\begin{ingredients}
  $\,$
  \begin{multicols}{2}
    \begin{itemize}
      \item $1/2$ red onion, cut into $1/8$-inch thick slices

      \item 1 Tbsb white wine vinegar plus extra for drizzling

      \item 2 cans cooked white beans

      \item 1 can tuna or however much you want

      \item 1 clove garlic

      \item 1 tsp Dijon mustard

      \item $1/4$ cup EVOO

      \item $1/4$ cup finely chopped parsley
    \end{itemize}
  \end{multicols}
\end{ingredients}
\begin{directions}
  $\,$
  \begin{enumerate}
    \item In a small bowl, soak red onion in cold water and agitate them, being careful not to break them. Let them sit for about 15 minutes, then drain water. Add vinegar and $1/4$ tsp salt and toss to coat. Set aside for 5 minutes.

    \item Combine beans and tuna in large bowl. Once the onions have marinated for 5 minutes, gently squeeze then dry and add to large bowl. Keep onion-vinegar-salt liquid in the smaller bowl.

    \item To the smaller bowl, add garlic,\footnote{Ideally, one should microplane the garlic directly into the acidic liquid already in the bowl.  Whatever the chemistry, getting raw garlic into a low-pH environment as quickly as possible after the cell walls are broken hinders the production of the compounds that make the harsh, raw tastes. An even better way is to mash the garlic in a mortar and pestle with the vinegar, but then one has to dirty a mortar and pestle.}
  \end{enumerate}
\end{directions}

\section{Tabbouleh}

\begin{ingredients}
  $\,$
  \begin{multicols}{2}
    \begin{itemize}
      \item $\frac{3}{4}$ lb (about 340g) ripe plum tomatoes, finely diced

      \item 2 cups finely chopped flat-leaf parsley

      \item $\frac{1}{4}$ cup dry bulgur wheat

      \item 1 cup finely chopped mint leaves

      \item 2 spring onions, finely chopped

      \item 5 Tbsp EVOO

      \item 2 Tbsp fresh squeezed lemon juice

      \item $\frac{1}{4}$ tsp ground coriander seed, and a pinch ground cinnamon
    \end{itemize}
  \end{multicols}
\end{ingredients}
\begin{directions}
  $\,$
  \begin{enumerate}
    \item Add tomatoes, together with 1 tsp of salt, to a strainer suspended over a bowl. Let drain for around 20 minutes, until about $\frac{1}{4}$ cup of liquid has collected in the bowl.

    \item Add parsley, together with 1 tsp of salt, to a second bowl, and let sit for 20 minutes. Blot with paper towels to dry.

    \item Boil $\frac{1}{2}$ cup of accumulated tomato juice, and add to the bulgur. Cover and let set approximately 30 minutes, until the bulgur has absorbed the tomato juice and become tender. Drail excess liquid.

    \item Combine
      \begin{itemize}
        \item tomatoes

        \item parsley

        \item mint

        \item bulgur

        \item scallions

        \item olive oil

        \item lemon juice

        \item coriander seed and cinnamon
      \end{itemize}
      and season.
  \end{enumerate}
\end{directions}

\section{Panzanella}
From \cite{seriouseatspanzanella}
\begin{ingredients}
  $\,$
  \begin{multicols}{2}
    \begin{itemize}
      \item 2.5 kg mixed tomatoes, roughly chopped

      \item 2 tsp kosher salt

      \item $3/4$ lb mixed

      \item 10 Tbsp evoo

      \item 1 small shallot, minced

      \item 2 medium cloves garlic, minced

      \item $1/2$ tsp dijon mustard

      \item 2 Tbsp white wine vinegar

      \item $1/2$ cup packed basil leaves, chopped
    \end{itemize}
  \end{multicols}
\end{ingredients}
\begin{directions}
  $\,$
  \begin{enumerate}
    \item Place tomatoes in a colander over a bowl, season with 2 tsp kosher salt, and toss to coat. Set aside at room temperature to drain, tossing occasionally.

    \item Meanwhile, preheat oven to $350^{\circ}$F. In a large bowl, toss bread cubes with 2 Tbsp evoo. Transfer to a baking sheet. Bake until firm and crisp but not browned, about 15 minutes. Remove from oven and let cool.

    \item Remove colander with tomatoes from bowl with tomato juice. Place colander and tomatoes in the sink. Add shallot, garlic, mustard, and vinegar to the bowl with tomato juice. Whisking, drizzle in the remaining $1/2$ cup olive oil to form an emulsification. Season dressing to taste with salt and pepper.

    \item Combine toasted bread, tomatoes, and dressing in a large bowl. Add basil leaves, and toss everything to coat. Season with salt and pepper. Let rest for 30 minutes.
  \end{enumerate}
\end{directions}
\begin{ingredients}
  $\,$
  \begin{multicols}{2}
    \begin{itemize}
      \item $\frac{3}{4}$ lb (about 340g) ripe plum tomatoes, finely diced

      \item 2 cups finely chopped flat-leaf parsley

      \item $\frac{1}{4}$ cup dry bulgur wheat

      \item 1 cup finely chopped mint leaves

      \item 2 spring onions, finely chopped

      \item 5 Tbsp EVOO

      \item 2 Tbsp fresh squeezed lemon juice

      \item $\frac{1}{4}$ tsp ground coriander seed, and a pinch ground cinnamon
    \end{itemize}
  \end{multicols}
\end{ingredients}
\begin{directions}
  $\,$
  \begin{enumerate}
    \item Add tomatoes, together with 1 tsp of salt, to a strainer suspended over a bowl. Let drain for around 20 minutes, until about $\frac{1}{4}$ cup of liquid has collected in the bowl.

    \item Add parsley, together with 1 tsp of salt, to a second bowl, and let sit for 20 minutes. Blot with paper towels to dry.

    \item Boil $\frac{1}{2}$ cup of accumulated tomato juice, and add to the bulgur. Cover and let set approximately 30 minutes, until the bulgur has absorbed the tomato juice and become tender. Drail excess liquid.

    \item Combine
      \begin{itemize}
        \item tomatoes

        \item parsley

        \item mint

        \item bulgur

        \item scallions

        \item olive oil

        \item lemon juice

        \item coriander seed and cinnamon
      \end{itemize}
      and season.
  \end{enumerate}
\end{directions}

\chapter{Sandwiches}
\section{Tuna melt}
\begin{ingredients}
  $\,$
  \begin{multicols}{2}
    \begin{itemize}
      \item 1 can tuna

      \item 2 Tbsp mayonnaise

      \item $\frac{2}{3}$ tsp smoked paprika

      \item $\frac{1}{3}$ tsp cayenne pepper

      \item 1 dash Worcestershire sauce

      \item 2 slices of good bread, about 1 inch thick

      \item $\frac{1}{2}$ cup grated cheddar
    \end{itemize}
  \end{multicols}
\end{ingredients}
\begin{directions}
  $\,$
  \begin{enumerate}
    \item Pre-heat oven to $325^{\circ}$ F.

    \item In a bowl, lightly mix tuna, mayonnaise, smoked paprika, cayenne, and Worcestershire sauce. Salt and pepper to taste. Leave medium-sized lumps of un-broken-up tuna---don't over-mix!

    \item Spoon on top of sliced bread and spread even. Cover with grated cheddar, and sprinkle over a shake of paprika for appearance.

    \item Bake for about 7 minutes, until cheese is melted.
  \end{enumerate}
\end{directions}
\chapter{Pasta}
Some general tips.
\begin{tip}\label{tip:starchypastawater}
  When you boil pasta, don't throw away the water. It's full of starch, and a great way to make oil-based or oil-heavy sauces cling to pasta when boiled down a bit.
\end{tip}

\section{Spaghetti all'arrabiatta}
\begin{ingredients}
  $\,$
  \begin{multicols}{2}
    \begin{itemize}
      \item 2 medium clove of garlic

      \item $\frac{1}{2}$ cup EVOO

      \item Roughly 1 tsp red pepper flakes

      \item 1 can tomatoes, broken up

      \item 2 anchovies

      \item $\frac{1}{2}$ tsp Worcestershire sauce

      \item 1 large serving of pasta
    \end{itemize}
  \end{multicols}
\end{ingredients}
\begin{directions}
  \begin{enumerate}
    \item In a large skillet, cook garlic and red pepper flakes in EVOO over medium low heat until the raw taste is gone, then add anchovy and cook an additional minute.

    \item When garlic is finished, start cooking pasta and stir in tomatoes to saucepan. Add Worcestershire sauce. Bring to a simmer.

    \item When pasta is cooked, transfer it to skillet. Turn heat to medium and cook an additional minute. Serve, garnishing with a glug of fresh olive oil.
  \end{enumerate}
\end{directions}

\section{Spaghetti aglio e olio}
\begin{ingredients}
  $\,$
  \begin{multicols}{2}
    \begin{itemize}
      \item 4 medium cloves of garlic, thinly sliced

      \item 1 large serving of pasta

      \item $\frac{1}{2}$ cup EVOO

      \item red pepper flakes
    \end{itemize}
  \end{multicols}
\end{ingredients}
\begin{directions}
  \begin{enumerate}
    \item In a large skillet, cook garlic and red pepper flakes in EVOO over medium low heat until lightly golden.

    \item Meanwhile, cook pasta in a small amount of water. Salt water fairly heavily.

    \item When garlic is cooked, transfer pasta and about $\frac{1}{2}$ cup of starchy water to skillet. Turn heat to high and toss rapidly to emulsify the starchy water and oil. Serve, garnishing with a glug of fresh olive oil.
  \end{enumerate}
\end{directions}

\section{Penne with blue cheese and mushrooms}
\begin{ingredients}
  \leavevmode
  \begin{multicols}{2}
    \begin{itemize}
      \item 1 glug EVOO

      \item 1 knob butter (optional)

      \item 1 container of mushrooms (ideally shiitake, but chestnut works too), sliced into thin strips

      \item 1 medium onion, thinly sliced

      \item One serving  fusili, penne, or similar

      \item 75g of blue cheese such as Farmhouse Blue
    \end{itemize}
  \end{multicols}
\end{ingredients}
\begin{directions}
  \begin{enumerate}
    \item Saute mushrooms with EVOO, butter, and coarsely ground black pepper on medium-high heat until they release their juices and begin to brown. Add onions and more pepper, and saut\'{e} until golden brown.

    \item Meanwhile, boil penne in salted water. When just shy of al dente, drain and add to onions and mushrooms, along with about 1 cup of the starchy water. Add blue cheese and stir vigorously to dissolve in the starchy water, then reduce until a creamy sauce is formed. Season with lots of pepper.
  \end{enumerate}
\end{directions}

\section{Puttanesca}From~\cite{seriouseatsputtanesca}.
\begin{ingredients}
  $\,$
  \begin{multicols}{2}
    \begin{itemize}
      \item 1 handful spaghetti

      \item 6 Tbsp EVOO

      \item 4 medium cloves garlic, thinly sliced

      \item 4--6 anchovies, minced

      \item red pepper flakes

      \item 1/4 cup capers, minced

      \item 1/4 cup sliced black olives

      \item 1 can peeled tomatoes

      \item 1 oz Parmesan cheeese, finely grated

      \item Parsley (optional)
    \end{itemize}
  \end{multicols}
\end{ingredients}
\begin{directions}
  $\,$
  \begin{enumerate}
    \item Boil pasta in as small an amount of water as possible\footnote{This concentrates the starch from the pasta, see \hyperref[tip:starchypastawater]{Tip~\ref*{tip:starchypastawater}}. There's a \emph{ton\/} of olive oil in this sauce.} with a pinch of salt

    \item Simultaneously, cook anchovies and garlic in EVOO over medium heat until garlic is lightly golden, then add capers and olives. Add tomatoes and simmer lightly.

    \item When pasta is slightly undercooked, add it, along with some pasta water, to the sauce. Cook, stirring, until past is done.

    \item Serve with parsley, EVOO, and Parmesan to garnish.
  \end{enumerate}
\end{directions}

\section{Pasta con le sarde}
From~\cite{foodwishespastaconlesarde}

This should leave the pan wetter than you want it, or the pasta will soak up all the moisture and develop the texture of stir fried noodles.
\begin{ingredients}
  $\,$
  \begin{multicols}{2}
    \begin{itemize}
      \item $\tfrac{1}{2}$ cup fresh or stale bread, crumbled into large breadcrumbs

      \item 1 handful spaghetti

      \item $\frac{1}{4}$ cup evoo

      \item $\frac{3}{4}$ cup yellow onion, finely diced

      \item $\frac{3}{4}$ cup fennel bulb, finely diced

      \item 2--3 medium cloves garlic, thinly sliced

      \item 1 anchovy fillet

      \item $\frac{1}{4}$ cup golden raisins

      \item 1 tsp tomato paste, for color

      \item $\frac{1}{4}$ cup wine

      \item 4 oz sardines (1 tin)

      \item $\frac{1}{8}$ cup toasted pine nuts

      \item Toasted breadcrumbs to garnish
    \end{itemize}
  \end{multicols}
\end{ingredients}
\begin{directions}
  $\,$
  \begin{enumerate}
    \item Toast pine nuts and reserve.

    \item Toast breadcrumbs in pan with olive oil until browned, then reserve to a bowl.

    \item Boil pasta water.

    \item Sauté onion, fennel, and red chili flakes in EVOO on medium heat with a generous pinch of salt until soft, about 10 minutes.

    \item Add anchovy, garlic, and golden raisins, and cook about 1 minute, then add wine and tomato paste, and reduce until wine has almost evaporated. Add about $\frac{1}{2}$ cup of boiling water and pine nuts.

    \item Cook pasta until just shy of al dente. Add sardines and break up, then stir in pasta.
  \end{enumerate}
\end{directions}

\section{Pasta Al Limone}
From~\cite{seriouseatspastaallimone}.
\begin{ingredients}
  $\,$
  \begin{multicols}{2}
    \begin{itemize}
      \item 5 Tbsp unsalted butter

      \item 1 medium lemon's worth of zest, plus juice

      \item 1 medium clove garlic, minced

      \item 2 handfuls

      \item 1 oz. Parmesan cheese, finely grated
    \end{itemize}
  \end{multicols}
\end{ingredients}
\begin{directions}
  $\,$
  \begin{enumerate}
    \item Melt butter in medium skillet over medium heat. Add lemon zest when butter begins to foam, cook 1--2 minutes, then remove from heat.

    \item Meanwhile, in a small amount of water, cook pasta with salt until not quite al dente.

    \item Add pasta and about 1 cup of starchy water to butter/zest mixture and cook over medium high heat.

    \item Add Parmesan and toss to combine. The pasta water, butter, and Parmesan should combine to create a creamy, emulsified sauce.

    \item Season with salt, pepper, and lemon juice ($\sim 1$ Tbsp) to taste.

    \item Serve, topping with more Parmesan and lemon zest.
  \end{enumerate}
\end{directions}

\section{Spaghetti al tonno}
From~\cite{foodwishesspaghettialtonno}. Serves 2 normal people, or one me.
\begin{ingredients}
  $\,$
  \begin{multicols}{2}
    \begin{itemize}
      \item 2 Tbsp EVOO

      \item red pepper flakes

      \item 3 anchovy fillets

      \item 1.5 Tbsp capers, optionally chopped

      \item 3 medium cloves garlic, minced

      \item $\frac{2}{3}$ cup dry white wime

      \item 1 can peeled tomatoes

      \item 1 can tuna, drained

      \item $\frac{1}{4}$ cup chopped italian parsley

      \item large handful (12 oz) spaghetti
    \end{itemize}
  \end{multicols}
\end{ingredients}
\begin{directions}
  \begin{enumerate}
    \item Boil pasta water.

    \item On medium heat, cook and red chili flakes and garlic in EVOO until raw smell is mostly gone, about 1 minutes. Add anchovies and capers, and stir until the anchovies break down and dissipate into the oil, about 2 minutes.

    \item Add wine, reduce by about $\frac{3}{4}$. Then add tomatoes and bring to a bare simmer. Add parsley and tuna, and season with salt.

    \item Add pasta to boiling water with a pinch of salt. Cook until just shy of al dente, then strain and add to sauce. Finish the pasta in the sauce for about a minute over medium-low heat; if the sauce is too dry to allow good mixing, add pasta water.

    \item Serve, garnishing with more parsley, and EVOO\@.
  \end{enumerate}
\end{directions}

\section{Spaghetti with meatballs}
Work in progress
\begin{ingredients}
  $\,$
  \begin{multicols}{2}
    \begin{itemize}
      \item 1 tbsp EVOO

      \item $\frac{1}{3}$ medium onion, very finely chopped

      \item $\frac{1}{2}$ tsp red chili flakes

      \item 3 medium cloves garlic

      \item 1--2 Tbsp tomato paste

      \item 1 can whole peeled tomatoes, crushed\footnote{Don't use canned crushed tomatoes. They are preserved with firming agents in order to keep the clumps of tomato intact, which isn't what we want here; one ends up with too-large, raw-tasting chunks.}

      \item 1 tsp dried basil

      \item 1 tsp dried oregano

      \item 2 meatballs from \hyperref[sec:meatballs_a_la_kurt]{Recipe~\ref*{sec:meatballs_a_la_kurt}}.

      \item 1 large serving of pasta (about $\frac{2}{5}$ of a bag/box)
    \end{itemize}
  \end{multicols}
\end{ingredients}
\begin{directions}
  \begin{enumerate}
    \item Saut\'{e} onion and chili flakes on medium low heat with around 1 Tbsp of salt until onion is browned and softened, about 7 minutes. Add garlic and saut\'{e} for about 2 additional minutes until the raw smell is gone, then turn up the heat to meduim high and add tomato paste. Cook about 5 minutes, then add tomatoes. Add basil and oregano and season with salt and pepper. Add meatballs if frozen and simmer for 10--20 minutes, depending on the canned tomatoes. Add almost cooked pasta and serve.
  \end{enumerate}
\end{directions}

\section{Pesto alla Genovese}
\begin{ingredients}
  $\,$
  \begin{multicols}{2}
    \begin{itemize}
      \item 2 medium cloves garlic

      \item 1-2 Tbsp pine nuts

      \item 1 pot's worth of fresh basil leaves, rinsed and spun until damp in a salad spinner

      \item roughly 1 cup grated parmesan

      \item 4 Tbsp olive oil
    \end{itemize}
  \end{multicols}
\end{ingredients}
\begin{directions}
  \begin{enumerate}
    \item Crush garlic in mortar and pestle (together with some coarse salt) until a paste forms, then add pine nuts and crush until oily. Add damp basil leaves and crush until the largest pieces of basil leaf are of order millimeters across. Add parmesan and combine, then add olive oil and blend until emulsified.

    \item To serve with pasta, \emph{do not} finish in sauce. Cook pasta and remove to mixing bowl, then add pesto and combine, adding a splash of pasta water if necessary to fully coat. As the mixed pasta sits the pasta will absorb moisture, leading to a dryer mixture.
  \end{enumerate}
\end{directions}


\section{Pasta with butternut squash}
From~\cite{foodwishesbutternutsquashpasta}.
\begin{ingredients}
  $\,$
  \begin{multicols}{2}
    \begin{itemize}
      \item 2 Tbsp EVOO

      \item 1 pound butternut squash (about $\frac{1}{2}$ large squash), cut into 1cm cubes

      \item 2 Tbsp unsalted butter

      \item 1 small shallot, finely minced

      \item 1 handful sage leaves or $\approx 1$ Tbsp dried sage if necessary

      \item 1 Tbsp juice from 1 lemon

      \item 1 lb pasta (ideally orecchiette)

      \item 1 oz Parmesan, finely grated
    \end{itemize}
  \end{multicols}
\end{ingredients}
\begin{directions}
  \begin{enumerate}
    \item Heat olive oil in large stainless steel skillet until barely smoking, then add squash. Season with salt and pepper. Cook until well-browned and squash is tender, about 10 minutes.\footnote{In the original recipe, it says this should ony take 5 minutes. This leaves the squash in cubes. I prefer it almost completely broken down.} Add butter and shallots and cook until butter is lightly browned and shallots are translucent, about 2 minutes. Add sage and stir in. Remove from heat and stir in lemon juice. Set aside.

    \item Cook pasta in a small amount of water with a pinch of salt until just shy of al dente. Drain pasta. Reserve starchy water.

    \item Add pasta and some starchy water to skillet and bring to a simmer over high heat. Cook until pasta is al dente, stirring and adding water as necessary. Remove from heat and stir in Parmesan and season with salt and pepper. Plate, garnish with more Parmesan, and serve.
  \end{enumerate}
\end{directions}

\section{Ricotta gnocchi}
From \cite{seriouseatsgnocchi}.
\begin{ingredients}
  \leavevmode
  \begin{multicols}{2}
    \begin{itemize}
      \item 340g Ricotta

      \item 30g grated parmesan

      \item 120g flour, plus some extra for dusting

      \item 1 whole egg plus 1 yolk
    \end{itemize}
  \end{multicols}
\end{ingredients}
\begin{directions}
  \begin{enumerate}
    \item Lay ricotta out onto paper towels and pat until towels have soaked up moisture. Repeat several times.

    \item Mix 250g dried ricotta, parmesan, flour, egg and egg yolk to bowl, and season with salt and pepper. Stir together, then kneading, add flour slowly until it becomes a workable dough.

    \item Transfer to a floured surface, roll into 8 logs 12 inches by $3/4$ inch.

    \item To cook, bring large pot of salted water to boil and add gnocchi. They are done when they float to the top.
  \end{enumerate}
\end{directions}

\chapter{Miscellaneous}

\section{Shakshuka}\label{sec:shakshuka}

Adapted from~\cite{seriouseatsshakshuka}.
\begin{ingredients}
  $\,$
  \begin{multicols}{2}
    \begin{itemize}
      \item 3 Tbsp EVOO (or ghee if you have some and you don't mind that it's not vegan)

      \item 1--2 medium onion, cut into 1--2cm chunks

      \item 2 red/orange/yellow bell pepper, cut into 1--2cm chunks

      \item 2--3 cloves garlic, thinly sliced

      \item 1$\frac{1}{2}$ Tbsp paprika (smoked or not, preferrably a mixture)

      \item 2 tsp Cumin

      \item $\frac{1}{2}$ tsp turmeric

      \item 1 can tomatoes

      \item parsley

      \item 2 eggs (traditional, but I usually don't use them so it's vegan)

      \item olives, feta, artichoke hearts

      \item Good bread
    \end{itemize}
  \end{multicols}
\end{ingredients}
\begin{directions}
  \begin{enumerate}
    \item Heat EVOO over high heat in a deep skillet until almost smoking, then add onions and peppers and cook, not stirring, until lightly charred. Stir and repeat a few more times, until well-cooked. When the developing fond threatens to burn, deglaze with a splash of water.

    \item Add garlic and cook $\sim$1 minute, until fragrant. Add spices (and more EVOO if dry) and cook about one minute, till infused into oil. Then add tomatoes and reduce heat to a bare simmer.

    \item Add some parsley and season with salt and pepper.

    \item (Optional) Break as many eggs you want (generally 2--4) on top, cover skillet, and simmer until cooked (internal temp. $63^{\circ}$C minimum.)

    \item Top with remaining parsley and olives, feta, and artichoke hearts.
  \end{enumerate}
\end{directions}


\section{Sausage, peppers, and onions}
\begin{ingredients}
  $\,$
  \begin{multicols}{2}
    \begin{itemize}
      \item 6 sausages

      \item 3 onions, chopped into $\sim$2cm squares

      \item 3 bell peppers, chopped into $\sim$2cm squares

      \item 1 Tbsp EVOO

      \item 1 Tbsp Herbes de Provence
    \end{itemize}
  \end{multicols}
\end{ingredients}
\begin{directions}
  \begin{enumerate}
    \item Brown sausages in saut\'{e} pan over high heat. Remove and cut into bite-sized pieces

    \item Turn heat to medium-high and add EVOO, peppers and onions. Cook until well-browned, then add Herbes de Provence and sausage.
  \end{enumerate}
\end{directions}

\section{Colcannon Hash}
\begin{ingredients}
  $\,$
  \begin{multicols}{2}
    \begin{itemize}
      \item Bacon, cubed

      \item About 3 large potates, cut into 1cm chunks

      \item $\frac{1}{2}$ cup of green onions, thinly sliced (mostly the white and light green parts)

      \item
    \end{itemize}
  \end{multicols}
\end{ingredients}
\begin{directions}
  \begin{enumerate}
    \item
  \end{enumerate}
\end{directions}

\section{Meatballs alla Kurt}\label{sec:meatballs_a_la_kurt}
\begin{ingredients}
  $\,$

  For the meatballs:
  \begin{multicols}{2}
    \begin{itemize}
      \item 500g ground beef

      \item 500g ground pork

      \item 1--2 eggs

      \item 5--6 springs thyme, picked clean

      \item 1 shallot, finely minced

      \item 3 cloves garlic, finely minced

      \item 1--2 tsp salt, plus some pepper

      \item Breadcrumbs, or ideally, a few pieces of bread soaked in water and squeezed out.
    \end{itemize}
  \end{multicols}
  For the `sauce':
  \begin{multicols}{2}

    \begin{itemize}
      \item A few more cloves of garlic and some whole star anise (optional)

      \item 2 red, orange, or yellow bell peppers, sliced into 5mm thick strips

      \item 3 onions, halved and sliced into 5mm thick strips

      \item A few handfuls of spinach
    \end{itemize}
  \end{multicols}
  For the rice:
  \begin{multicols}{2}
    \begin{itemize}
      \item As much risotto rice as is necessary, $\frac{1}{2}$ cup per serving

      \item A knob of butter

      \item $\frac{1}{4}$ tsp turmeric per $\frac{1}{2}$ cup rice

      \item Salt and pepper to taste

      \item A few handfuls of spinach
    \end{itemize}
  \end{multicols}
\end{ingredients}
\begin{directions}
  \begin{enumerate}
    \item Mix all meatball ingredients together, cover, and let sit in refrigerator for 30 minutes.

    \item On low heat, put (optionally) a few cloves of garlic and the star anise into a heavy skillet with some olive oil, and let them infuse into the oil

    \item Form large meatballs, remove garlic and star anise from skillet, and sear on at least two sides until dark brown. Then add bell peppers, onions, and salt to draw out moisture. Lower heat to medium, and let simmer until the peppers and onions have broken down. Add spinach and let wilt, season with salt and pepper.

    \item Meanwhile, melt butter in saucepan and add rice, turmeric, and salt and pepper. Cook until rice becomes slightly translucent, then slowly add water, maintaining a thick consistency.
  \end{enumerate}
\end{directions}

\section{Mushroom rag\`{u}}\label{sec:mushroom_ragu}

From \cite{seriouseatsmushroomragu}

\begin{ingredients}
  $\,$

  \begin{multicols}{2}
    \begin{itemize}
      \item 1 oz (about 30g) dried porcini mushrooms

      \item 3.5 lb (about 1.6 kg) assorted fresh mushrooms, thinly sliced

      \item 4 Tbsp EVOO

      \item 2 medium onions, minced

      \item 1 large carrot, minced

      \item 15 cloves garlic, minced

      \item 1 cup dry white wine

      \item 1 can whole tomatoes

      \item 4 sprigs thyme
    \end{itemize}
  \end{multicols}
\end{ingredients}
\begin{directions}
  \begin{enumerate}
    \item In a bowl, combine porcini mushrooms with 2 cups boiling water and let sit for 15 minutes. Drain, squeeze mushrooms dry, and slice, reserving mushroomy water.

    \item In an enormous pot, cook onion, carrot, and garlic with EVOO over high heat until soft, about 6 minutes. Then add add fresh and soaked mushrooms and cook everything until mushrooms are sticking to the bottom of the pot and threatening to burn; this can take close to half an hour. The darker they get, the better, but if they burn you have to start over. Deglaze with wine and mushroom liquid, then add tomatoes and thyme.

    \item Lower heat to medium low and cook for 1-2 hours, stirring occasionally.
  \end{enumerate}
\end{directions}

\section{Beef, beer, and bean chili}\label{sec:beef_beer_and_bean_chili}
From \cite{foodwishesbeefbeanbeerchili}
\begin{ingredients}
  $\,$

  \begin{multicols}{2}
    \begin{itemize}
      \item 1 Tbsp vegetable oil

      \item 1 diced onion

      \item 1 diced green pepper

      \item 2 lb ground beef

      \item 1 tsp ground black pepper

      \item 3 Tbsp chili powder

      \item 1 Tbsp ground cumin

      \item $\frac{1}{8}$ tsp ground cinnamon

      \item 2 tsp paprika

      \item 1 tsp unsweetend cocoa powder

      \item $\frac{1}{4}$ tsp dried oregano

      \item $\frac{1}{4}$ tsp dried cayenne

      \item 3 cloves garlic, minced

      \item 1 bottle good beer such as a dark ale

      \item 1 can crushed tomatoes

      \item 1 can chicken stock

      \item 2 cans pinto beans, rinced well
    \end{itemize}
  \end{multicols}
\end{ingredients}
\begin{directions}
  \begin{enumerate}
    \item Add onion, green pepper, and beef to large pot along with salt. Cook stirring over high heat until liquid evaporates and meat is well browned and beginning to form a fond, about 10 minutes. Turn down heat to medium and add black pepper, chili powder, cumin, cinnamon, paprika, and garlic. Cook about 4 minutes, until fond is threatening to burn, then deglaze with beer. Reduce by about $50\%$, then add tomatoes and stock. Simmer for 30-45 minutes, add beans and green pepper, and simmer another 30 minutes. Skim off fat if it's there.
  \end{enumerate}
\end{directions}



\section{Generic chicken curry}
\label{sec:chicken_tikka_masala}

\begin{ingredients}

  \leavevmode
  Spice blend:
  \begin{multicols}{2}
    \begin{itemize}
      \item 2 tsp kosher salt

      \item 1 tsp ground turmeric

      \item 2 tsp garam masala

      \item 2 tsp ground cumin

      \item 1 tsp ground coriander

      \item $\tfrac{1}{8}$ tsp ground cardamon

      \item $\tfrac{1}{2}$ tsp black pepper

      \item $\tfrac{1}{8}$ tsp Cayenne pepper

      \item $1$ tsp smoked paprika
    \end{itemize}
  \end{multicols}
  Everthing else:
  \begin{multicols}{2}
    \begin{itemize}
      \item $1\tfrac{1}{8}$ lb boneless skinless chicken thighs

      \item 1 Tbsb vegetable oil

      \item 2--3 Tbsp clarified butter

      \item 1 medium onion, chopped

      \item $\frac{1}{4}$ cup tomato paste

      \item 4 garlic cloves, finely grated

      \item 1 rounded Tbsp finely grated peeled ginger

      \item 1 can crushed tomatoes

      \item 1 can coconut milk

      \item 1 cup chicken broth

      \item $\frac{1}{2}$ tsp crushed red pepper flakes

      \item 2 Tbsp freshly chopped cilantro

      \item A few wedges of lime
    \end{itemize}
  \end{multicols}
\end{ingredients}
\begin{directions}
  \begin{enumerate}
    \item Combine chicken, spice blend, and vegetable oil in a bowl, cover, and refrigerate for 30 minutes.

    \item In a sauté pan on high heat, brown chicken thighs, then reserve to a bowl. Turn heat to medium high, add onion, and sauté until just translucent. Add tomato paste, and saute until caramelizing, about 6 minutes. Add garlic and ginger, cook one minute, and deglaze with crushed tomatoes. Turn heat to medium low and add coconut milk and chicken stock. Let simmer for 15 minutes.

    \item Cut up chicken thighs into bite sized pieces and add to sauté pan. Reduce until sauce is thick, add fresh cilantro. Serve over rice and garnish with cilantro and a wedge of lime.
  \end{enumerate}
\end{directions}

\section{Channa masala}
\label{sec:channa_masala}

\begin{ingredients}

  \leavevmode
  Spice blend:
  \begin{multicols}{2}
    \begin{itemize}
      \item 2 tsp kosher salt

      \item 1 tsp ground turmeric

      \item 2 tsp garam masala

      \item 1 tsp ground coriander

      \item $\tfrac{1}{8}$ tsp ground cardamon

      \item $\tfrac{1}{2}$ tsp black pepper
    \end{itemize}
  \end{multicols}
  Everthing else:
  \begin{multicols}{2}
    \begin{itemize}
      \item 1 knob ginger, about 1 inch long

      \item 4 garlic cloves, finely grated

      \item 1 lemon

      \item 1 tsp vegetable oil

      \item 1 Tbsp clarified butter

      \item 1 tsp whole cumin seed

      \item 2 tsp whole mustard seeds

      \item 1 medium onion, finely diced

      \item 1 can whole tomatoes

      \item 1 rounded Tbsp finely grated peeled ginger

      \item 1 can crushed tomatoes

      \item 2 cans chickpeas

      \item $\frac{1}{2}$ cup cilantro
    \end{itemize}
  \end{multicols}
\end{ingredients}
\begin{directions}
  \begin{enumerate}
    \item In a mortar and pestle, grind garlic, ginger, 1 Tbsp lemon juice, and 1 tsp coarse salt until fully broken down.

    \item Heat clarified butter over medium high heat in a large saucepan until shimmering. Add black mustard and cumin seeds and cook about 15 seconds, then add onion and salt. Cook on high heat, stirring vigorously and adding water when a fond develops or onions threaten to burn, until onions are deep brown, about 10 minutes.

    \item Lower heat to medium and immediately add garlic, ginger, and chili paste. Add spice blend and stir until fragrant. Add tomatoes and crush. Add chickpeas and cilantro

    \item Bring to a simmer, and cook about 30 minutes. If it is getting too thick, add lid.

    \item Stir in more lemon juice to taste
  \end{enumerate}
\end{directions}

\begin{appendices}
  \chapter{Osmosis and the virial theorem}
  \label{ch:the_virial_theorem}

  In this appendix we derive the standard equation relating osmosis pressure to the concentration of solutes. This derivation comes, mutatis mutandis, from \cite{whatisosmosis}.

  We consider a system of $N$ particles moving inside a box of volume $V$ with position vectors $x_{i} \in \R^{3}$ and momentum vectors $p_{i} \in \R^{3}$. Consider the quantity
  \begin{equation*}
    Q = \sum_{i} x_{i} \cdot p_{i},
  \end{equation*}
  known as the \emph{virial.}
  The time derivative of this quantity is
  \begin{align}
    \tder{Q}{t} &= \sum_{i} \dot{x}_{i} \cdot p_{i} + \sum_{i} x_{i} \cdot \dot{p}_{i} \\
    \label{eq:derivative_of_virial}
    &= 2 T + \sum_{i} \dot{x}_{i} \cdot F_{i},
  \end{align}
  where $T$ is the total kinetic energy of the system, and $F_{i}$ is the net force on the $i$th particle.

  The particles are confined to some finite volume, so there exists some $L \in \R$ such that $\abs{x_{i}} < L$. If we make the reasonable simplifying assumption that the norms of the momenta of the particles are also bounded above by some constant $P$, then the virial is bounded above by some real number $Q_{\mathrm{max}} = NLP$; this is certainly a reasonable assumption if our particles are the molecules in a container of water sitting on a table. Then the time average of the time derivative of the virial at very large times is
  \begin{align}
    \lim_{\tau \to \infty} \left|\left\langle \tder{Q}{t} \right\rangle_{\tau}\right| &= \lim_{\tau \to \infty} \left| \frac{1}{\tau} \int_{0}^{\infty} \tder{Q}{t} \, \d \tau\right| \\
    &= \lim_{\tau \to \infty} \left| \frac{Q(\tau) - Q(0)}{\tau} \right| \\
    &\leq \lim_{\tau \to \infty} \frac{2Q_{\mathrm{max}}}{\tau} \\
    \label{eq:time_average_vanishes}
    &= 0.
  \end{align}

  Taking the time average of \hyperref[eq:derivative_of_virial]{Equation~\ref*{eq:derivative_of_virial}} and using  yields the \emph{virial theorem:}
  \begin{theorem}[virial theorem]
    For a system of $N$ particles whose positions $x_{i}$ and momenta $p_{i}$ are bounded, at large times $\tau \gg 0$ we have the asymptotic relation
    \begin{equation}
      \label{eq:virial_theorem}
      0 = 2\langle T \rangle_{\tau} + \left\langle \sum_{i} \dot{x}_{i} \cdot F_{i}\right\rangle_{\tau}.
    \end{equation}
  \end{theorem}
\end{appendices}

We continue our analysis by noting that the force $F_{i}$ has two contributions: one from the particles' interactions with the walls of our box, and one from their interactions with each other. We write
\begin{equation*}
  F_{i} = F_{i}^{\mathrm{wall}} + F_{i}^{\mathrm{particle}}.
\end{equation*}
The first term is, in the continuum limit, responsible for the macroscopic pressure of our liquid on the walls of the container; the second is responsible for the internal dynamics of the liquid. Plugging this decomposition into \hyperref[eq:virial_theorem]{Equation~\ref*{eq:virial_theorem}} yields
\begin{equation*}
  -2\langle T \rangle_{\tau} = \left\langle \sum_{i} \dot{x}_{i} \cdot F_{i}^{\mathrm{wall}} \right\rangle + \left\langle \sum_{i} \dot{x}_{i} \cdot F_{i}^{\mathrm{particle}} \right\rangle
\end{equation*}
We now take the continuum limit of the first term. The force is zero except when the particles are at the wall of the box, and on the wall of the box it is proportional


%%fakechapter Bibliography
\printbibliography{}

\end{document}
